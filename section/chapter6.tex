\chapter{Syarah Fauziatul Ulya/1144075}
\begin{table}[h]
\caption{Tabel Kegiatan Harian}
\centering
\begin{tabular}{|c|c|c|}
\hline
No&Tanggal&Kegiatan\\
\hline
1.&\multirow{2}{*}{25-02-2019}&1) Membuat repositori Modul Praktikum kelas Pemrograman III.\\
&&2) Menginstal python, pip, anaconda, dan jupiter dari kursus Udemy.\\
\hline
2.&\multirow{2}{*}{26-02-2019}&1) Membuat repositori Modul Praktikum kelas Kecerdasan Buatan.\\
&&2) Mendata dan menilai tugas 1 kelas 2A- Pemrograman III.\\
&&3) Membimbing anak kelas 3C untuk tugas Kecerdasan Buatan.\\
\hline
3.&\multirow{2}{*}{27-02-2019}&1) Mendata dan menilai tugas kelas 3C-Kecerdasan Buatan.\\
&&2) Membimbing anak kelas 3C untuk tugas Kecerdasan Buatan.\\
&&3) Mengkoordinasi tugas kontribusi anak kelas.\\
\hline
\end{tabular}
\label{table:contoh}
\end{table}

\begin{table}[]
\caption{Tabel Laporan Harian (Kamis, 28 Februari 2018)}
\resizebox{\textwidth}{!}{%
\begin{tabular}{|l|l|l|}
\hline
\textbf{No} & \multicolumn{1}{c|}{\textbf{Kategori}} & \multicolumn{1}{c|}{\textbf{Keterangan}} \\ \hline
1 & Dedikasi & BukuInformatika/flask \#12 \\ \hline
2 & Produktifitas & \begin{tabular}[c]{@{}l@{}}a. Evaluasi mingguan dan sosialisasi format baru penilaian \\ peserta internship II di IRC dan Prodi.\end{tabular} \\
 &  & \begin{tabular}[c]{@{}l@{}}b. Memeriksa lembar jawaban UAS GIS kelas 3C dan Arkom \\ kelas 1C. \end{tabular}\\
 &  & \begin{tabular}[c]{@{}l@{}}c. Menginput nilai UAS kelas 1C, 3A, 3B, dan 3C di google\\ docs. \end{tabular} \\
 &  & \begin{tabular}[c]{@{}l@{}l@{}}d. Mengkoordinir mahasiswa untuk tugas kontribusi\\ pembuatan cover, pencetakan, sampai pendistribusian buku \\di Grup Whatsapp.\end{tabular} \\
 &  & \begin{tabular}[c]{@{}l@{}l@{}}e. Memeriksa tugas kecerdasan buatan kelas 3C serta meng-\\input nilai ke google docs atas nama Fadila, Lusia Violita \\Aprilian, dan Rahmi Roza.\end{tabular} \\ \hline
3 & Integritas & able to merge/has no conflict \\ \hline
4 & Disiplin & Jam Datang : 07.55 WIB \\
 &  & Jam Pulang : 17.00 WIB \\ \hline
5 & Loyalitas & \begin{tabular}[c]{@{}l@{}}Menyapu, membersihkan dan merapihkan meja, mencuci\\ gelas nomor 6, dan mengecek AC di pagi dan sore hari.\end{tabular} \\ \hline
\end{tabular}%
}
\end{table}

\begin{table}[]
\caption{Tabel Laporan Harian (Kamis, 28 Februari 2018)}
\resizebox{\textwidth}{!}{%
\begin{tabular}{|l|l|l|}
\hline
\textbf{No} & \multicolumn{1}{c|}{\textbf{Kategori}} & \multicolumn{1}{c|}{\textbf{Keterangan}} \\ \hline
1 & Dedikasi &  \\ \hline
2 & Produktifitas & \begin{tabular}[c]{@{}l@{}}a. Mengerjakan Soal Toefl Pre-Test 2 (Computer Based)\\Skor Toefl: 55\end{tabular} \\
 &  & \begin{tabular}[c]{@{}l@{}}b. Mendata skor toefl peserta Internship II di IRC.\end{tabular}\\
 &  & \begin{tabular}[c]{@{}l@{}}c. Memeriksa pull request di laporanirc/2019 \end{tabular}\\ \\ \hline
3 & Integritas & able to merge/has no conflict \\ \hline
4 & Disiplin & Jam Datang : 07.39 WIB \\
 &  & Jam Pulang : 14.20 WIB \\ \hline
5 & Loyalitas & \begin{tabular}[c]{@{}l@{}}Menyapu, membersihkan dan merapihkan meja, membeli\\ sabun dan spons cuci piring, dan mengecek AC\\ di pagi dan sore hari.\end{tabular} \\ \hline
\end{tabular}%
}
\end{table}